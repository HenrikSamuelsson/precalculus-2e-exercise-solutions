% Minimal sample solution (style-separated)
% Compile with: tectonic solution-openstax-min.tex
\documentclass[a4paper,11pt]{article}
\usepackage{../../exercise-style-cmbright}

% Metadata (adjust per exercise)
\renewcommand{\BookTitle}{Precalculus 2e (OpenStax, 2021)}
\renewcommand{\ExerciseSource}{Section}
\renewcommand{\SectionLabel}{1.1}
\renewcommand{\ExerciseNumber}{5}
\renewcommand{\AuthorName}{Henrik Samuelsson}
\renewcommand{\DateStamp}{2025-09-03}
\renewcommand{\ProblemTitle}{Exercise Solution}

\begin{document}
\maketitle

\begin{problem}
What is the difference between a relation and a function?
\end{problem}

\begin{solution}
A \emph{relation} is any set of ordered pairs \((x,y)\). Its \emph{domain} is the set of all first coordinates, and its \emph{range} is the set of all second coordinates.

A \emph{function} is a special kind of relation in which each input \(x\) is paired with \emph{exactly one} output \(y\). Equivalently, no two different ordered pairs in the set share the same first coordinate. Graphically, a function passes the \emph{vertical line test} (every vertical line intersects the graph at most once).

Examples:
\[
R=\{(1,2),(1,3),(2,4)\}\quad\text{is a relation but \emph{not} a function (the input }1\text{ has two outputs).}
\]
\[
F=\{(1,2),(2,4),(3,6)\}\quad\text{is a function (each input has exactly one output).}
\]

Thus, every function is a relation, but not every relation is a function.
\end{solution}

\begin{answer}
A relation is a set of ordered pairs. A function is a relation in which no two ordered pairs have the same first coordinate.
\end{answer}

\end{document}
